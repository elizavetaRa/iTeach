\documentclass[a4paper,10pt]{article}
\usepackage[utf8]{inputenc}
\usepackage[default,osfigures,scale=0.95]{opensans}\usepackage{graphicx}
\usepackage[T1]{fontenc}
\usepackage{graphicx}
\usepackage{epsfig}
\usepackage{german}
\usepackage{url}
\usepackage{fancyhdr}
\hyphenation{me-cha-nik}

\pagestyle{fancy}
\fancyhf{}
\fancyhead[L]{
	\includegraphics[scale=0.1]{./figures/logo.png}
}
%\fancyhead[C]{}
\fancyfoot[R]{\thepage}
\setlength{\headheight}{20pt}
\setlength{\parindent}{0pt}

\begin{document}

{\bfseries \large Protokoll vom 07. Mai 2015 \\[1mm]		%Kann durch Überschrift ersetzt werden
\normalfont Autor: Philipp Plotz}					%Autor sollte immer angegeben werden

\paragraph{}
Wir trafen uns gemeinsam als Gruppe mit unserem Tutor zum wöchentlichen Tutorentreffen um unseren Zwischenstand zu präsentieren, die weitere Vorgehensweise abzusprechen und die Aufgaben für die nächste Abgabe zu bekommen.\\
Zuerst musste die Frage über die zu verwendende Programmiersprache geklärt werden, bei welcher es noch sehr geteilte Meinungen innerhalb der Gruppe gab. Einerseits stand HTML5 andererseits Flash zur Debatte. Da jedes seine Vor- und Nachteile mit sich bringt fiel die Entscheidung nicht leicht. Doch aufgrund des erheblichen Mehraufwands bei der Benutzung von HTML5 entschieden wir uns nach einer Abstimmung welche zu einem 1:5 Ergebnis führte, für den einfacheren Weg mit Flash. Des weiteren hat unser Tutor uns über eine didaktisch notwendige Funktion des Spiels aufgeklärt, welche wir bis zum jetzigen Zeitpunkt komplett außer acht gelassen hatten. Der Spieler darf nicht erst zum Ende des Spiels auf seine Fehler in der seinem Verhalten hingewiesen werden, sondern während des Spiels hier und da Hinweise bekommen, wie er sich richtig zu verhalten hat. Für dessen Umsetzung dachten wir an eine Art Buch (angelehnt an den echten Knigge) welches der Protagonist kann und sich so Tipps für die kommenden Aufgaben geben lassen kann.
\paragraph{}
Inhalt der nächsten Abgabe (bis 17. Mai):\\
Erste Überlegungen zu dem finalen Layout machen und diese zu Papier bringen (PDF-Datei). Diese Layoutideen sollen eine grobe Richtung zeigen, welches Erscheinungsbild das Spiel gegen Ende haben soll. Dieses Vorablayout wird am 20. Mai in der Mediendidaktik Vorlesung vor den anderen Gruppen präsentiert. Ein Sprecher, welcher unsere Gruppe vertritt wurde noch nicht festgelegt.
\paragraph{}
Gruppeninterne Aufgabenstellung (bis 10. Mai):\\
Eigene Layoutentwürfe ins Git pushen. Diese sollten Ideen für das Layout der Startseite, einiger Minigames und für das Ingame Spielerlebnis beinhalten.

\end{document}