\documentclass[a4paper,10pt]{article}
\usepackage[utf8]{inputenc}
\usepackage[default,osfigures,scale=0.95]{opensans}\usepackage{graphicx}
\usepackage[T1]{fontenc}
\usepackage{graphicx}
\usepackage{epsfig}
\usepackage{german}
\usepackage{url}
\usepackage{fancyhdr}
\hyphenation{me-cha-nik}

\pagestyle{fancy}
\fancyhf{}
\fancyhead[L]{
	\includegraphics[scale=0.1]{./figures/logo.png}
}
%\fancyhead[C]{}
\fancyfoot[R]{\thepage}
\setlength{\headheight}{20pt}
\setlength{\parindent}{0pt}

\begin{document}

{\bfseries \large Protokoll vom 27.04.2015 \\[1mm]		%Kann durch Überschrift ersetzt werden
\normalfont Autor: Philipp Plotz}					%Autor sollte immer angegeben werden

\paragraph{Teilziele:}								%Kann durch kurze Beschreibung ersetzt werden
Eigene Ideen für das Uni-Knigge Lernspiel.
Einigung auf Kommunikations- und Entwicklungsplattformen und erweiterte gruppeninterne Strukturierung.\\


Bei unserem ersten Treffen mit dem Gruppentutor haben wir die groben Inhaltspunkte des Projektablaufs erfahren und uns über den aktuellen Stand ausgetauscht. Desweiteren wurden die wesentliche Spielinhalte besprochen und erste Gedanken darüber innerhalb der Gruppe debattiert. Es wurde darüber diskutiert wie die Inhalte des Spiels (didaktischer Hintergrund) dem Spieler möglichst effektiv übermittelt werden.

\paragraph{Aufgaben bis zum nächsten Abgabetermin} 
\begin{itemize}
	\item Erstellung eines Zeitplans
	\item Spielgeschichte detaillierter formulieren.
	\item Ablaufdiagramm (Spielstruktur) erstellen
	\item Zielgruppe konkretisieren
	\item Ziel des Spiels ausformulieren (Welches Wissen soll vermittelt werden?)
	\item Minispiele für die einzelnen Situationen/Stationen des Spiels überlegen
	\item Corporate Design (Logo, Formatvorlagen) fertig haben und ins git laden
\end{itemize}

\paragraph{Weiteres}
Das Spiel muss auf dem Browser Firefox lauffähig sein. \\
Wir legten einen Zeitpunkt für unser Gruppentreffen fest. Das Treffen wird Donnerstags 11 Uhr in der Fakultät für Informatik stattfinden.

\end{document}