\documentclass[a4paper,10pt]{article}
\usepackage[utf8]{inputenc}
\usepackage[default,osfigures,scale=0.95]{opensans}\usepackage{graphicx}
\usepackage[T1]{fontenc}
\usepackage{graphicx}
\usepackage{epsfig}
\usepackage{german}
\usepackage{url}
\usepackage{fancyhdr}
\hyphenation{me-cha-nik}

\pagestyle{fancy}
\fancyhf{}
\fancyhead[L]{
	\includegraphics[scale=0.1]{./figures/logo.png}
}
%\fancyhead[C]{}
\fancyfoot[R]{\thepage}
\setlength{\headheight}{20pt}
\setlength{\parindent}{0pt}

\begin{document}

{\bfseries \large Protokoll vom 21. Mai 2015 \\[1mm]		%Kann durch Überschrift ersetzt werden
\normalfont Autor: Philipp Plotz}					%Autor sollte immer angegeben werden

\paragraph{}
Wir trafen uns als Gruppe mit unserem Gruppentutor um den aktuellen Stand und das weitere Vorgehen zu Besprechen.\\
Zu Beginn berieten wir uns darüber in wie weit wir auf Herr Prof. Groh's Ratschläge eingehen und haben die allgemeine Gestaltung unsere Spiels überdacht. Dabei entschieden wir uns dafür, dass das 'Karlchen'-Logo nun nicht mehr den Startbildschirm zieren soll, sondern nur in der Spielhilfe auf dem Smartphone wiederzufinden ist.\\Des Weiteren mussten wir uns Gedanken darüber machen, den Kontrast bei der Abschlusspräsentation zu verbessern, da bei der gestrigen Zwischenpräsentation das Bild schwierig zu erkennen war. Jedoch optimieren wir den Kontrast nur für die Präsentation, sprich den Beamer, da unser Spiel am Bildschirm bereits ausreichend gute Kontraste bietet.\\Wir sprachen außerdem über die allgemeine Hauptspielbewegung, ob sie nun wie von an Anfang geplant per Point-And-Click Verfahren gesteuert werden soll oder einer minimalistisch animierten Steuerung ähnlich dem Spiel 'Pokémon'. Wir entschieden uns aufgrund der einfacheren Umsetzung für die erste Variante.\\Die Spielstruktur durch welche der Spieler geführt wird, wird einen semilinearen Ablauf aufweisen. Das heißt, dass es nicht komplett irrelevant ist für den weiteren Spielverlauf, welche Entscheidungen der Spieler trifft.\\Da es Fragen zur Programmiersprache ActionScript 3 gab, klärten wir diese mit dem Tutor auf ein hinreichendes Maß ab.\\Der Button 'Entwickler' im Hauptmenü wird umbenannt in 'Credits'

\paragraph{Aufgaben bis 07. Juni 2015\\}
Abgabe eines Prototypen welcher die Hauptseite, das Smartphone und mind. ein Minispiel beinhaltet. Das Wichtigste dabei ist, dass es lauffähig ist. Auf die Gestaltung wird vorerst weniger geachtet.


\end{document}