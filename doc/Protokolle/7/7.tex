\documentclass[a4paper,10pt]{article}
\usepackage[utf8]{inputenc}
\usepackage[default,osfigures,scale=0.95]{opensans}\usepackage{graphicx}
\usepackage[T1]{fontenc}
\usepackage{graphicx}
\usepackage{epsfig}
\usepackage{german}
\usepackage{url}
\usepackage{fancyhdr}
\hyphenation{me-cha-nik}

\pagestyle{fancy}
\fancyhf{}
\fancyhead[L]{
	\includegraphics[scale=0.1]{./figures/logo.png}
}
%\fancyhead[C]{}
\fancyfoot[R]{\thepage}
\setlength{\headheight}{20pt}
\setlength{\parindent}{0pt}

\begin{document}

{\bfseries \large Protokoll vom 10. Juni 2015 \\[1mm]		%Kann durch Überschrift ersetzt werden
\normalfont Autor: Philipp Plotz}					%Autor sollte immer angegeben werden

\paragraph{}
Wir trafen uns geschlossen als Gruppe mit unserem Tutor.\\
Unsere bisherige Arbeit wurde dabei vom Tutor kritisch beurteilt. Das Intro, welches den Spieler eine Einführung in die Geschichte geben soll und ihn ins Spiel führen soll, ist sehr gut gelungen. Bei dem Minispiel 'Busspiel' kam die Frage auf, inwiefern eine Rückmeldung an den Spieler erfolgt, wenn dieser eine der anderen Personen berührt. Allgemeine Hinweise waren dahingehend, dass noch einige kleine Programminkonsistenzen (Bugs) behoben werden müssen. Deswegen besteht die Frage, welche wir noch nicht beantworten konnten, ob die Zahlen beim Timer groß oder eher klein gehalten werden sollen. Die Creditseite, welche zur Zeit noch als Testseite dient muss logischerweise noch geändert werden.\\
Bei der Abschlusspräsentation muss das Spiel nicht unbedingt /gespielt/ werden. Es reicht wenn eine Slideshow von Screenshots gezeigt wird oder das Spiel bis zu einem bestimmten Punkt schon /vorgespielt/ wurde. Die Präsentation an sich soll nicht länger als 7 Minuten dauern. Wir legten folgende Einteilung für die Präsentation fest:
\begin{itemize}

        \item Niklas Präsentationshalter
        \item Lisa Jury (beurteilt Projekte anderer Gruppen)

\end{itemize}
\end{document}