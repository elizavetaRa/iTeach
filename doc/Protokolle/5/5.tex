\documentclass[a4paper,10pt]{article}
\usepackage[utf8]{inputenc}
\usepackage[default,osfigures,scale=0.95]{opensans}\usepackage{graphicx}
\usepackage[T1]{fontenc}
\usepackage{graphicx}
\usepackage{epsfig}
\usepackage{german}
\usepackage{url}
\usepackage{fancyhdr}
\hyphenation{me-cha-nik}

\pagestyle{fancy}
\fancyhf{}
\fancyhead[L]{
	\includegraphics[scale=0.1]{./figures/logo.png}
}
%\fancyhead[C]{}
\fancyfoot[R]{\thepage}
\setlength{\headheight}{20pt}
\setlength{\parindent}{0pt}

\begin{document}

{\bfseries \large Protokoll vom 20. Mai 2015 \\[1mm]		%Kann durch Überschrift ersetzt werden
\normalfont Autor: Philipp Plotz}					%Autor sollte immer angegeben werden

\paragraph{}
Wir gingen geschlossen als Gruppe zur Lehrveranstaltung Vorlesung Mediendidaktik in der vierten Doppelstunde. In dieser Vorlesung hat jede Gruppe die Aufgabe ihr Projekt den anderen Gruppen in Anwesenheit von Prof. Friedrich und Prof. Groh zu präsentieren. Jede Gruppe musste intern eine Person bestimmen, welche den Vortrag halten soll. Da wir uns pflichtbewusst schon im Vornherein Gedanken über unsere Präsentation gemacht haben, lief alles reibungslos ab. Niklas meldete sich bereits freiwillig als Vortragender. Die Stichpunkte für die Präsentation hat Lisa zusammengetragen. Als Präsentationsmedium wurde uns das Dateiformat PDF vorgegeben in welchem die ersten Entwurfsideen für das Endprodukt als Screenshots aufgeführt sind. Nach dem gehaltenen Vortrag stellte Herr Prof. Friedrich folgende Fragen:\\\\
Was ist an unserem Spiel anders als das Spiel aus dem Vorjahr, welches ein ähnliches Thema behandelt?
\begin{itemize}
	\item Das Spiel vom Vorjahr war auf den Stoff des ersten Semesters bezogen, wohingegen unseres sich mehr auf soziale Kompetenzen bezieht und unsere Wissensvermittlung eher eine Art Gewissens- und Organisationsfrage ist.
\end{itemize} 
Was ist der Lerneffekt unseres Spiels?
\begin{itemize}
	\item Ein strukturierter (zeitlicher) Ausgleich aus sozialen Kompetenzen und Lernwilligkeit. Oft fällt es Studenten im ersten Semester schwer in den  selbst organisierten Alltag einzusteigen. Wir wollen ihnen im Spiel die Möglichkeiten und ihr Konsequenzen vor Augen führen.
\end{itemize}
Herr Prof. Groh’s einziger Einwand zu unserer Spielgestaltung war, dass unser Logo nicht in den Gesamtstil des Spiels hineinpasst und daher am Besten weggelassen werden sollte.

\end{document}