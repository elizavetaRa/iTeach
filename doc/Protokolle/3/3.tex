\documentclass[a4paper,10pt]{article}
\usepackage[utf8]{inputenc}
\usepackage[default,osfigures,scale=0.95]{opensans}\usepackage{graphicx}
\usepackage[T1]{fontenc}
\usepackage{graphicx}
\usepackage{epsfig}
\usepackage{german}
\usepackage{url}
\usepackage{fancyhdr}
\hyphenation{me-cha-nik}

\pagestyle{fancy}
\fancyhf{}
\fancyhead[L]{
	\includegraphics[scale=0.1]{./figures/logo.png}
}
%\fancyhead[C]{}
\fancyfoot[R]{\thepage}
\setlength{\headheight}{20pt}
\setlength{\parindent}{0pt}

\begin{document}

{\bfseries \large Protokoll vom 29. April 2015 \\[1mm]		%Kann durch Überschrift ersetzt werden
\normalfont Autor: Philipp Plotz}					%Autor sollte immer angegeben werden

\paragraph{}
Wir trafen uns geschlossen als Gruppe um Einzelheiten über unser Lernspiel zu besprechen und um unsere Ideen auszutauschen. Folgende Ergebnisse kamen zustande.\\
Wir einigten uns darauf, dass unsere Spielfigur (Protagonist) ein männlicher Student der Informatik sein soll. Der Spieler soll durch einen durchschnittlichen Tag als Student an der Technischen Universität Dresden geführt werden. Angefangen mit dem Aufstehen über die Fahrt mit der Buslinie 61 zum TU Campus, Besuchen von Kursen bestimmter Module, bis hin zu studentischen Abendaktivitäten wie das Nachgehen von Sportangeboten. Noch nicht einig geworden sind wir uns darüber, ob der Spieler einen Tag erleben soll oder sich über eine komplette Woche hin beweisen muss. Des weiteren konnten wir uns noch nicht sicher für oder gegen die Idee entscheiden ob der Spieler eine Art Konto besitzen soll, auf welchem er Geld sparen kann, welches er dann einsetzen kann um Boni zu erhalten. An Geld kommt der Spieler indem er Mate-Eistee Falschen sammelt und abgibt oder als Studentische Hilfskraft (SHK) einem Job nachgeht (zB. Kellner). Geld soll aber nicht primär der Ansporn sein bei unserem Lernspiel, da dies uns aus didaktischen Gründen nicht zusagt. Viel mehr soll der Spieler über zwei Punktzahlen (Scores) verfügen. Der eine Score symbolisiert den Beliebtheitsgrad des Spielers in seinem Umfeld, der Andere hingegen den Wissensstand. Diese Scores könnten als Balkengrafik am Spielfensterrand angezeigt werden. Ebenfalls auf dem Spielbildschirm angezeigt werden, soll eine Uhr, welche den Spielfortschritt anzeigt. Dem Spieler soll als Spielhilfe ein Smartphone zur Verfügung stehen, über welches er Nachrichten von Kommilitonen erhält. Diese Nachrichten enthalten wichtige Informationen über den weiteren Spielverlauf. Am Ende des Spiels, soll eine Auswertung über das Verhalten des Spielers erfolgen. Er wird in eine Personengruppe eingeteilt (Äußerst sozialer Mitstudent, Einzelgänger, ...) und daraufhin Empfehlungen ausgesprochen, wie er sich besser in seinem Umfeld verhalten könnte.\\
\paragraph{Minispielideen}
\begin{itemize}
	\item Busfahrt: Gleichgewichtsspiel
	\item Vorlesung/Übung finden: Labyrinth
	\item Vorlesung: Stoff wird vermittelt (keine Interaktion)
	\item Übung: Vorlesungsstoff wird abgefragt von Kommilitonen.
	\item Mensa: Essen auffangen (Kommilitone hat Essen fallen lassen)
\end{itemize}

\paragraph{}
Ein erster Entwurf des Ablaufdiagramms wurde erstellt.


\end{document}