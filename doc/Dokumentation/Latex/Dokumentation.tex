%%%%%%%%%%%%%%%%%%%%%%%%%%%%%%%%%%%%%%%%%
% Thin Sectioned Essay
% LaTeX Template
% Version 1.0 (3/8/13)
%
% This template has been downloaded from:
% http://www.LaTeXTemplates.com
%
% Original Author:
% Nicolas Diaz (nsdiaz@uc.cl) with extensive modifications by:
% Vel (vel@latextemplates.com)
%
% License:
% CC BY-NC-SA 3.0 (http://creativecommons.org/licenses/by-nc-sa/3.0/)
%
%%%%%%%%%%%%%%%%%%%%%%%%%%%%%%%%%%%%%%%%%

%----------------------------------------------------------------------------------------
%	PACKAGES AND OTHER DOCUMENT CONFIGURATIONS
%----------------------------------------------------------------------------------------

\documentclass[a4paper, 11pt]{article} % Font size (can be 10pt, 11pt or 12pt) and paper size (remove a4paper for US letter paper)

\usepackage[protrusion=true,expansion=true]{microtype} % Better typography
\usepackage{graphicx} % Required for including pictures
\usepackage{wrapfig} % Allows in-line images

\usepackage{mathpazo} % Use the Palatino font
\usepackage[T1]{fontenc} % Required for accented characters
\linespread{1.05} % Change line spacing here, Palatino benefits from a slight increase by default

\usepackage[utf8]{inputenc}
\usepackage[colorlinks, pdfpagelabels, pdfstartview = FitH, bookmarksopen = true, bookmarksnumbered = true, linkcolor = black, plainpages = false, hypertexnames = false, citecolor = black] {hyperref}
\usepackage{hyphenat}

\makeatletter
\renewcommand\@biblabel[1]{\textbf{#1.}} % Change the square brackets for each bibliography item from '[1]' to '1.'
\renewcommand{\@listI}{\itemsep=0pt} % Reduce the space between items in the itemize and enumerate environments and the bibliography

\renewcommand{\maketitle}{ % Customize the title - do not edit title and author name here, see the TITLE block below
\begin{flushright} % Right align
{\LARGE\@title} % Increase the font size of the title

\vspace{50pt} % Some vertical space between the title and author name

{\large\@author} % Author name
\\\@date % Date

\vspace{40pt} % Some vertical space between the author block and abstract
\end{flushright}
}

%----------------------------------------------------------------------------------------
%	TITLE
%----------------------------------------------------------------------------------------

\title{\textbf{Dokumentation des Medida Projekts "Uniknigge"}\\ % Title
von Hung Tran Duc, Elizaveta Ragosina, Philipp Plotz, Christoph Jurkowski, Niklas Fallik und Sheyda Hayatgheybi} % Subtitle

\author{\textsc{Gruppe 1 (Tutorin: Bianca Preißler)} % Author
\\{\textit{Technische Universität Dresden}}} % Institution

\date{16.07.2015} % Date

%----------------------------------------------------------------------------------------

\begin{document}

\maketitle % Print the title section

%----------------------------------------------------------------------------------------
%	ABSTRACT AND KEYWORDS
%----------------------------------------------------------------------------------------

\renewcommand{\abstractname}{Präambel} % Uncomment to change the name of the abstract to something else

\vspace{5cm} % Some vertical space between the abstract and first section


\begin{abstract}
\noindent
Die hier vorliegende Dokumentation des Mediendidaktik und -psychologie Praktikums der Gruppe 1 mit Hung Tran Duc, Elizaveta Ragosina, Philipp Plotz, Christoph Jurkowski, Niklas Fallik und Sheyda Hayatgheybi legt die Vorstellung des Spiels, die technische Umsetzung, die Evaluation mit der Zielgruppe, den Projektverlauf
und das eigene Fazit sowie ein Statement von jedem Gruppenmitglied dar. Das Lernspiel "Uniknigge" basiert auf dem weitverbreiteten Adobe Flash und bildet zusammen mit diversen Mini-Spielen und Grafiken ein vollständiges Spiel.
\end{abstract}

% \hspace*{3,6mm}\textit{Keywords:} lorem , ipsum , dolor , sit amet , lectus % Keywords

\newpage 
\section*{Historie}
Wichtige Änderungen und zugehörige ausführende Autoren.\\
\begin{tabular}{|lllp{5.5cm}|}
\hline 
\textbf{Version} & \textbf{Datum} & \textbf{Autor(en)} & \textbf{Bemerkungen} \\ 
\hline 
0.1 & 16.07.2015 & Christoph Jurkowski & \nohyphens{Erstellung der Struktur, Präambel, Einfügen von Dokumentationsteilen} \\ 
\hline
\end{tabular} 


%----------------------------------------------------------------------------------------
%	ESSAY BODY
%----------------------------------------------------------------------------------------

\newpage
\renewcommand{\contentsname}{Inhaltsverzeichnis}
\tableofcontents

\newpage
\section{Einführung}
Diese Dokumentation besschreibt die Umsetzung des Lernspiels "Uniknigge" von der Idee bis zum vollständigen Spiel durch die Entwickler und Grafiker Hung Tran Duc, Elizaveta Ragosina, Philipp Plotz, Christoph Jurkowski, Niklas Fallik und Sheyda Hayatgheybi.

\section{Vorstellung des Spiels}
\includegraphics[scale=0.5]{images/vorstellung.png}\\\\
Unsere Grundidee war es ein Lernspiel zu entwickeln, welches als universalen Verhaltenskodex für angehende Studenten am Beispiel eines Informatikstudenten der TU Dresden dient. Angelehnt an den Verhaltenskodex haben wir unser Spiel „Uniknigge“ genannt. Unsere Zielgruppe sind angehende Studenten, die kurz vor dem Antritt ihres Erststudiums sind, d.h. zwischen der Immatrikulation und dem ersten Tag in der Universität. Dabei benötigt der Spieler keinerlei spezifische Vorkenntnisse, sondern soll intuitiv nach bestem Gewissen agieren. \\

Das Spiel selbst führt den Spieler durch einen fiktiven ersten Tag in der Uni und stellt ihn immer wieder vor soziale wie auch intellektuelle Herausforderungen. Die Herausforderungen werden in der Spielstory wie auch in den Minispielen offensichtlich oder versteckt gestellt. Während des Intros wird dem Spieler in Form einer Smartphone App eine ausführliche Erklärung, welche man jederzeit im Spiel wieder aufrufen kann, zur Verfügung gestellt. Zusätzlich werden vor jedem Minispiel, weitere Regeln zum Verhalten wie auch zur Spielanwendung bereitgestellt. Nach den Minispielen sowie bei Herausforderungen während des Spiels, wird eine kurze Auswertung gezeigt, ggf. mit Lob und Kritik. Am Ende des Spiels folgt eine längere Resolution mit Verweisen auf jeweilige Herausforderungen mit Verbesserungsvorschlägen für den nächsten Spielablauf. Da unser Spiel eine Verhaltensanleitung darstellt, ist das Ziel und die Zielfindung zunächst nicht klar erkennbar und wird erst während des Spiels sichtbar.
Das Ziel ist es, eine Balance zwischen dem aufmerksamen, leistungsfokussierten Lernen und den für das Studium überlebenswichtigen Sozialkompetenzen zu erreichen.

\section{Technische Umsetzung}
\includegraphics[scale=0.5]{images/umsetzung.png}\\\\
\subsection{Layout}
\subsection{Allgemeines}
\subsection{Hauptmenü}
Unser Hauptmenü, wie auch alle weiteren Spielsequenzen haben als Hintergrund ein Foto, das in Photoshop bearbeitet, an eine Comic-ähnliche Aufmachung erinnert. Diesen Stil haben wir auch auf die Figuren und sämtliche andere Gegenstände übertragen.
Zusätzlich ist im Hauptmenü der Schriftzug „Uniknigge“ und drei Buttons zu sehen. Die Buttons sind zum Starten des Spiels, Fortsetzen des Spiels und zur aktuellen Punkteabfrage vorgesehen. Das Hauptmenü dient der Einstimmung des Spielers auf das Spiel und die Spielumgebung.

\subsection{Intro}
Wie bereits erwähnt, zieht sich unser Layout aus Comic-ähnlichem Bildmaterial durch das ganze Spiel. Das Intro ist eine kleine Filmsequenz, der den Spieler in die Problematik und Aufgabenstellung, wie auch Anwendungen beispielsweise das Smartphone, einführt. Zu sehen ist ein Student, der vor seinem ersten Unitag nicht schlafen kann. Um ihm die Angst zu nehmen, bekommt er eine Kurznachricht. Auf seinem Smartphone kann er so Regeln, Anweisungen und Tipps einsehen.

\subsection{Spielablauf}
Der Spielablauf ist als "Point-and-Click“ Anwendung konzipiert. Entscheidungsfragen, häufig eingeleitet durch einen virtuellen Kommilitonen, werden in Form von Sprechblasen dargestellt.
Auch die Punkteübersicht und der zeitliche Fortschritt des gesamten Spiels sind zu jeder Zeit in Form von Balken bzw. einer Uhr sichtbar.

\subsection{Minispiele}
Bezüglich des Designs sind die Minispiele im gewohnten Comic-Layout gehalten, doch im Gegensatz zu dem herkömmlichen Spielablauf werden hierfür die Pfeiltasten zur Bedienung bemüht. Das hat den Effekt, dass der Spieler merkt, dass ihm bei diesen Aufgaben mehr abverlangt wird, als im sonstigen Spielablauf. Zudem kann er die Maus benutzen, um die Anwendung zu pausieren.

\subsection{Smartphone}
Das Smartphone dient als Informationsquelle. Zu jedem Zeitpunkt während des Spiels bietet es dem Spieler die Möglichkeit Fakten und Verhaltensregeln im Allgemeinen oder auch zur spezifischen Situation nachzulesen.

\subsection{Outro}
Das Outro ist an das Intro angelehnt. Der Student befindet sich nach seinem ersten Tag wieder in seinem Bett, wo er seinen ersten Tag an der Uni Revue passieren lässt. Er bekommt über seine Smartphone-App die Evaluation mit Tipps und Anmerkungen zu seinen Entscheidungen. 

\subsection{Hilfsmittel für die Entwicklung}

\subsection{Hilfsmittel für die Kommunikation}



\section{Evaluation mit der Zielgruppe}
\includegraphics[scale=0.5]{images/evaluation.png}\\\\
Das Lernspiel gibt angehenden Studenten, die zwischen der Immatrikulierung und der ersten Vorlesungszeit stehen die Möglichkeit, den Unialltag spielerisch in seiner Vielseitigkeit kennenzulernen. Ausgerichtet auf diesen Anspruch haben wir die Bedienelemente, Texte und Aufgaben auf ein hohes Niveau gehoben, damit sich die angehenden Studenten damit identifizieren können. Neben dem spielerischen Aspekt ist es denkbar, dass das Lernspiel auch auf den Webseiten von Universitäten und Berufsorientierungsportalen zu Verfügung gestellt werden kann. Abiturienten können mit diesem Spiel auf Fragen wie zum Beispiel "Wie sieht der Unialltag aus?“, "Wie finde ich mich an der Uni zurecht?“ oder "Wie komme ich in Kontakt mit anderen Studenten?“ Antworten finden. Um die Spielatmosphäre während des Spielens und des Lernens etwas aufzulockern, haben wir fiktive Elemente, wie Karlchen, der dem Spieler nachts die Smartphone-App empfiehlt, eingebaut. Außerdem sind die Minispiele überzogene Anlehnungen an unsere eigenen Erfahrungen an das erste Semester. Ein Beispiel, das jeder Student kennt, ist das Labyrinth zur Raumfindung. 
\section{Projektverlauf}
\includegraphics[scale=0.5]{images/projektverlauf.png}\\\\
\section{Fazit}
\includegraphics[scale=0.5]{images/projektverlauf.png}\\\\

\section{Statement von jedem Gruppenmitglied}
\includegraphics[scale=0.5]{images/statement.png}\\\\
\subsection{Hung Tran Duc}
\subsection{Elizaveta Ragosina}
Durch das Praktikum konnte man eigene Erfahrung im Bereich Spielentwicklung mit Adobe Flash machen. Da ich mit anderen Adobe Produkten schon vertraut war und am Lernen der Flash-Funktionsweisen interessiert war, war meine Einarbeitungszeit vergleichsweise gering. Das Designen hat auch sehr viel Spaß gemacht und ich konnte viele neue Erfahrungen mit den verwendeten Programmen sammeln. Zwar konnte ich in der Gruppe das Konsistenzkonzept, aufgrund von unterschiedlichen Vorstellungen und Zeitmangel für unser Spiel nicht vollständig durchsetzen, dennoch gefällt mir das Endprodukt sehr.
Noch zu erwähnen ist, dass unsere Gruppenarbeit bei weiten nicht perfekt war, was teilweise an längerer Einarbeitungszeit, teilweise an fehlender Motivation einiger Gruppenmitglieder lag. Das führte dazu, dass ein Teil der Gruppe einen großen Anteil der Aufgaben Anderer übernehmen mussten und so sehr viel mehr zum Endergebnis beigetragen haben. Mein Zeitaufwand lag ungefähr beim Vier- bis Fünffachem vom Vorgesehenem und war damit dem Zeitaufwand beim SWT-Praktikum aus dem 2. Semester ähnlich. Ich denke auch, wenn alle Gruppenmitglieder gleich viel beitragen würden, wäre der Zeitaufwand für dieses Spiel höher als der Vorgesehene. Aus meiner Sicht ist es umso so mehr Schade, das der Einfluss auf die Modulnote in keinem Vergleich zur aufgewendeten Zeit steht.
Alles in allem war das Projekt sehr lehrreich. Die Spielentwicklung hat mich begeistert. Ich kann mir vorstellen, mein Studium im Bereich der Spielentwicklung zu vertiefen.

\subsection{Philipp Plotz}
Das Praktikum bot mir eine gute Gelegenheit einen Einblick in die Flash Programmierung mit ActionScript3 zu bekommen, welchen ich sonst aus eigenen Stücken nicht vorgenommen hätte. Es war sehr interessant und erstaunlich, wie einfach sich komplexere Mechanismen implementieren ließen und es machte auch nach kurzer Einarbeitungszeit auch Spaß. Leider konnten durch Zeitmangel nicht alle Ideen umgesetzt und der ganze Wille eingebracht werden, was nicht heißen soll, dass das Praktikum zu umfangreich gestaltet ist. Des Weiteren wurden die Arbeitsbedingungen durch intragruppale Konflikte, welche durch unterschiedliche Ansichten vom Endprodukt hervorgerufen wurden, nicht gerade erleichtert. Summa summarum hat sich die Arbeit aus meiner Sicht gelohnt, da ein recht ansehnliches Produkt entstanden ist, welches man gern auch anderen Personen präsentiert. Das Spiel ist konsistent umgesetzt und bietet aus meiner Sicht keine möglichen Kritikpunkte, da allen Anforderungen genüge getan ist.
\subsection{Christoph Jurkowski}

\subsection{Niklas Fallik}
Die Aufgabenstellung habe ich im Hinblick auf den Schwierigkeitsgrad angemessen gefunden.
Die Zwischenpräsentation des Designs war ebenso aufschlussreich wie die ausführlichen Konsultationen unerer Tutorien und die Evaluation mit der Zielgruppe.
In dem Praktikum konnte ich bereits Erlerntes aus den vergangenen Semestern anwenden.
Das zur Umsetzung des Projekts genutzte Adobe Flash mit ActionScript 3.0 bedurfte einer geringen Einarbeitungszeit, sodass ich mit der Impelentierung der mir zugeteilten Aufgaben schnell beginnden konnte.
Zukünftigen Projektgruppen dieses Praktikums würde ich empfehlen, sich einen Team-internen Zeitplan zu erarbeiten und sich streng an diesen zu halten.

\subsection{Sheyda Hayatgheybi}

%----------------------------------------------------------------------------------------

\end{document}