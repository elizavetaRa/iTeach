	\documentclass[a4paper,10pt]{article}
\usepackage[utf8]{inputenc}
\usepackage[condensed,sfdefault]{universalis}
\usepackage{graphicx}
\usepackage{epsfig}
\usepackage{german}
\usepackage{url}
\usepackage{fancyhdr}
\hyphenation{me-cha-nik}

\pagestyle{fancy}
\fancyhf{}
\fancyhead[L]{
	\includegraphics[scale=0.1]{./figures/logo.png}
}
%\fancyhead[C]{}
\fancyfoot[L]{Erstellt am: \today}
\fancyfoot[R]{\thepage}
\setlength{\headheight}{20pt}
\setlength{\parindent}{0pt}

\begin{document}

\vspace*{1cm}

{\bfseries \large Uniknigge \\[1mm]		%Kann durch Überschrift ersetzt werden
\normalfont Autor: Niklas Fallik}					%Autor sollte immer angegeben werden

\vspace{1cm}

\begin{abstract}					
	Es soll ein Lernspiel entstehen, speziell für Erstsemester konzipiert, das dem Lernenden auf einfache Art richtige Verhaltensweisen im Unialltag nahe bringen soll.
\end{abstract}
\vspace{1cm}

Folgende Szenarien sind denkbar:

Kategorie I (Organisatorisches):
\begin{itemize}
	\item Umgang mit dem Online-Einschreibesystem Jexam, evtl. in Anlehnung an dessen Demo-Video
	\item Aktivieren/Aufladen der Mensa-Karte
	\item Was ist zu tun im Krankheitsfall (während der Vorlesungszeit und Prüfungszeit)
	\item ...
\end{itemize}

Kategorie II (Knigge):
\begin{itemize}
	\item Verhalten in den Warteschlangen in der Mensa
	\item Verhalten in der Straßenbahn bzw. im Bus auf dem Weg von und zur Uni, evtl. in Anlehnung an
die \glqq Spielplatz\grqq-Seite der DVB
	\item Verhalten in Übungen, Vorlesungen
	\item Verhalten gegenüber Kommilitonen in Gruppenarbeiten
	\item Verhalten gegenüber Kommilitonen mit höherer sexueller Anziehungskraft (Flirt-Tipps)
	\item Verhalten gegenüber Dozenten, Professoren und sonstigen Lehrbeauftragten
	\item ...
\end{itemize}

Umsetzung:\\
Im Hintergrund der Lernspielanwendung sollen typische Bilder aus den jeweiligen Umgebungen der Szenarien gezeigt werden. Der Eindruck, dass der Lernende teil des Geschehens ist, könnte dadurch verstärkt werden typische Umgebungsgeräusche (Tellerklirren in der Mensa, Motorengeräusch aus dem Bus) einzubinden.\\
Ferner könnte ein Balken am Bildschirmrand (ähnlich dem Balken einer Lebensanzeige in Shootern) den Beliebtheitsgrad des Lernenden in der virtuellen Uniwelt widerspiegeln. Ziel des Spiels ist es dann, diesen Balken auf 100\% zu bringen. Für jede erfüllte Aufgabe gibt es je nach Schwierigkeitsgrad Punkte, die dem Bliebtheitspunktekonto gutgeschrieben werden.
Diese Aufgaben sollen aus Minigames bestehen.\\
Denkbar ist z.B. das Durchqueren eines Labyrinths in der Mensa mit einem vollen Tablett, wobei der Lernende anderen Besuchern der Mensa ausweichen muss.\\
Ein weitere realisierbare Aufgabenstellung an den Lernenden wäre das Einprägen des Gesichts eines Kommilitonen, das er dann in einer vorgegebenen Zeit in einem vollbesetzten Hörsaal wieder erkennen muss.

\end{document}