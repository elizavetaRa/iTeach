	\documentclass[a4paper,10pt]{article}
\usepackage[utf8]{inputenc}
\usepackage[default,osfigures,scale=0.95]{opensans}\usepackage{graphicx}
\usepackage{graphicx}
\usepackage{epsfig}
\usepackage{german}
\usepackage{url}
\usepackage{fancyhdr}
\hyphenation{me-cha-nik}

\pagestyle{fancy}
\fancyhf{}
\fancyhead[L]{
	\includegraphics[scale=0.1]{./figures/logo.png}
}
%\fancyhead[C]{}
\fancyfoot[L]{Erstellt am: \today}
\fancyfoot[R]{\thepage}
\setlength{\headheight}{20pt}
\setlength{\parindent}{0pt}

\begin{document}

\vspace*{1cm}

{\bfseries \large Plot Uniknigge \\[1mm]		%Kann durch Überschrift ersetzt werden
\normalfont Autor: Niklas Fallik}					%Autor sollte immer angegeben werden

\vspace{1cm}

\begin{abstract}					
	Protagonist ist ein männlicher Informatikstudent im ersten Semester. Der Spieler (im folgenden auch als Lernender bezeichnet) wird durch Alltagssituationen auf dem Campus der TU Dresden geführt.
\end{abstract}
\vspace{1cm}

\paragraph{Roter Faden}
	Der Wecker klingelt, der Protagonist steht auf. Mit der Buslinie 61 muss er zu seiner ersten Veranstaltung gelangen. Später wird er zum Mittagessen in die Mensa gehen, anschließend soll er eine Übung besuchen. Nach seinen Verpflichtungen kann er Sportangebote wahrnehmen oder studentischen Abendaktivitäten nachgehen.

\paragraph{Werkzeug}
	Dem Lernenden soll ein Smartphone als Hilfsmittel während der gesamten Spieldauer zur Verfügung stehen, das nur einen Mausklick entfernt ist. Es dient als Kommunikationsmittel, über das der Spieler von seinen virtuellen Kommilitonen erreicht werden soll, um den Spielfortschritt voranzutreiben. Während des Spiels soll es nützliche Informationen wie einen Kalender mit den anstehenden Terminen im Spiel bereit halten.

\paragraph{Motivation}
	Dem Lernenden werden während des Spiels zwei Spielstandsbalken (Scores) angezeigt. Für seine Entscheidungen erhält er je nach Tragweite Punkte, die sich auf sein Beliebtheitsgrad auswirken, oder auf sein Wissensstand. Eine Uhr am Bildschirmrand soll den Spielfortschritt angeben.

\paragraph{Ziel}
	Am Ende eines Spieldurchgangs soll der Lernende ein Feedback zu seinem Sozialverhalten auf dem Campus bekommen. Er wird auf Grundlage der Entscheidungen, die er während des Spiels getroffen hat, in eine Personengruppe eingeteilt. So erhält der Spieler einen Gesamteindruck, wie er selbst möglicherweise auf sein Umfeld wirkt. Daraufhin werden ihm Empfehlungen ausgesprochen, wie er sich künftig besser verhalten könnte.

\paragraph{Umsetzung}
	Die folgenden Minispiele sind als Herausforderungen für den Lernenden in Planung:
	\begin{itemize}
		\item Gleichgewichtsspiel auf Busfahrten
		\item Labyrinth zum Zurechtfinden auf dem Campus, Finden zu den Lehrveranstaltungen
		\item Auffangen von heruntergefallenem Essen in der Mensa
		\item Kleines Sportspiel
	\end{itemize}
	Darüber hinaus soll der Lernende immer wieder vor Entscheidungen stehen, die als Grundlage für die Bewertung seines Verhaltens dienen soll. Sie können sich auf den Gesamtspielverlauf auswirken, vor allem spiegeln sie sich allerdings in den Spielstandsbalken – Beliebtheitsgrad und Wissensstand – wieder.

\end{document}